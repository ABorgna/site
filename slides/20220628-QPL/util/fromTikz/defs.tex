% Math notation

\newcommand{\bigo}{\ensuremath{\mathcal{O}}}
\newcommand{\Dn}[1]{\ensuremath{\mathcal{D}_{#1}}} % Set of n-qubit density matrices
\newcommand{\N}{\ensuremath{\mathbb{N}}}
\newcommand{\R}{\ensuremath{\mathbb{R}}}
\newcommand{\W}{\ensuremath{\mathbb{W}}}
\newcommand{\Ztwo}{\ensuremath{\mathbb{Z}_2}}
\newcommand{\Ftwo}{\ensuremath{\mathbb{F}_2}}
\newcommand{\one}{\ensuremath{\mathbbm{1}}}
\newcommand{\diag}{\ensuremath{\mathscr{D}}} % Set of diagrams
\ifdefined\C
  \renewcommand{\C}{\ensuremath{\mathbb{C}}}
\else
  \newcommand{\C}{\ensuremath{\mathbb{C}}}
\fi
\ifdefined\U
  \renewcommand{\U}{\mathtt{U}}
\else
  \newcommand{\U}{\mathtt{U}}
\fi
\ifdefined\R
  \renewcommand{\R}{\mathtt{R}}
\else
  \newcommand{\R}{\mathtt{R}}
\fi
\newcommand{\ketbra}[2]{\ket{#1}\!\!\bra{#2}}
\renewcommand{\braket}[2]{\langle#1\mid#2\rangle}
\newcommand{\imod}[2]{{#1}\ \text{mod}\ {#2}} % integer modulo, n mod k
\newcommand{\idiv}[2]{{#1}\ \text{div}\ {#2}} % integer div, n div k

\DeclareMathOperator{\rank}{rank}
\DeclarePairedDelimiter\abs{\lvert}{\rvert}

% Types and terms
\newcommand{\bang}[1]{!^{#1}}
\newcommand{\meas}{\mathtt{meas}}
\newcommand{\new}{\mathtt{new}}
\newcommand{\bit}{\mathtt{B}}
\newcommand{\qubit}{\mathtt{Q}}
\renewcommand{\unit}{\mathtt{Unit}}
\newcommand{\nat}{\mathtt{Nat}}
\renewcommand{\vec}[2]{\mathtt{Vec}\ {#2}\ {#1}}
\newcommand{\Qlet}[3]{\mathtt{let\ }{#1}={#2}\mathtt{\ in\ }{#3}}
\newcommand{\Qfold}{\mathtt{fold}}
\newcommand{\Qmap}{\mathtt{map}}
\newcommand{\Qaccumap}{\mathtt{accuMap}}
\newcommand{\Qsplit}{\mathtt{split}}
\newcommand{\Qappend}{\mathtt{append}}
\newcommand{\Qcompose}{\mathtt{compose}}
\newcommand{\Qfor}[3]{\mathtt{for\ }{#1}\mathtt{\ in\ }{#2}\mathtt{\ do\ }{#3}}
\newcommand{\ifz}[3]{\mathtt{ifz\ }{#1}\mathtt{\ then\ }{#2}\mathtt{\ else\ }{#3}}
\newcommand{\vnil}{\mathtt{VNil}}
\newcommand{\Qrange}{\mathtt{range\ }}
\newcommand{\lambdaD}{\ensuremath{\lambda_D}}
\newcommand{\Qdrop}{\mathtt{drop}}

\newcommand{\trans}[1]{\left\llbracket{#1}\right\rrbracket}
\newcommand{\eval}[1]{\left\lfloor{#1}\right\rfloor}
\newcommand{\dual}[1]{{#1}^\bot}
\newcommand{\types}[1]{types({#1})} % For the list of types in a context

% Categories
\newcommand{\interpret}[1]{\left\{\!\!\!\left\{{#1}\right\}\!\!\!\right\}}
\newcommand{\CPMo}{\ensuremath{\overline{\mathbf{CPMs}}^\oplus}}
\newcommand{\CPMs}{\textbf{CPMs}~}
\newcommand{\CPM}{\textbf{CPM}~}
% Symbols

\usetikzlibrary{circuits.ee.IEC}
\usetikzlibrary{shapes.gates.logic.US,shapes.gates.logic.IEC}

\newcommand{\ground}{%
    \begin{tikzpicture}[circuit ee IEC,yscale=1.0,xscale=1.0]
    \draw[solid,arrows=-] (0,1ex) to (0,0) node[anchor=center,ground,rotate=-90,xshift=.66ex] {};
    \end{tikzpicture}
}

%\usetikzlibrary{cd}
%\newcommand{\bvdots}{%
%  \tikz[baseline, every node/.style={inner sep=0}]{%
%    \node at (0,0){.}; \node at (0,-3pt){.}; \node at (0,3pt){.};
%  }
%}
\newcommand{\sground}{\scalebox{0.5}{\!\ground}}

% ZX diagrams

\newcommand{\zxGND}{\ensuremath{\text{ZX}_{\ground}}}
\newcommand{\szxGND}{\ensuremath{\text{SZX}_{\ground}}}
\newcommand{\zxB}{\ensuremath{\text{ZX}^!}}

% Matrix size annotations
\newcommand{\annoted}[3]{{\scriptstyle~#1}\left\lbrace\mathrlap{\phantom{#3}}\right.\overbrace{#3}^{#2}}

% - Note making

\definecolor{orange}{RGB}{200,127,0}
\newcommand{\TODOtxt}[1]{{\color{red}TODO:~\color{darkgray}#1}\vspace*{0.5em}}
\newcommand{\NOTEtxt}[1]{{\color{orange}NOTE:~\color{darkgray}#1}\vspace*{0.5em}}
\newcommand{\citationneeded}{\textsuperscript{\color{red} [Citation needed] }}
\newcommand{\TODO}[1]{\pdfmargincomment[icon=Insert,color=red]{#1}}
\newcommand{\NOTE}[1]{\pdfmargincomment[icon=Note,color=orange]{#1}}
\newcommand{\NOTEab}[1]{\pdfmargincomment[icon=Note,color=purple,author=AB]{#1}}
\newcommand{\NOTErr}[1]{\pdfmargincomment[icon=Note,color=brown,author=RR]{#1}}

% ------ Theorems, lemmas and things -------------
%\numberwithin{sect}{section}% [if desired]
%\numberwithin{equation}{section}
%\theoremclass{Theorem}
%\theoremstyle{plain}
%\theorembodyfont{\upshape}
%\newtheorem{thm}[equation]{Theorem}
%
%\theoremstyle{plain}
%\theorembodyfont{\upshape}
%\newtheorem{lemma}[equation]{Lemma}
%\newtheorem*{lemmaN}[equation]{Lemma} % non-numbered
%\newtheorem{prop}[equation]{Proposition}
%\newtheorem{cor}[equation]{Corollary}
%
%%\newtheoremstyle{plainN} % Plain style, without numbering
%%%{\item[\hskip\labelsep \theoremheaderfont {##1}\theoremseparator]}
%%{\item[\hskip{##1}]}
%%{\item[\hskip\labelsep \theoremheaderfont {##1}\ {##3}\theoremseparator]}
%%\theoremstyle{plainN}
%%\newtheorem{thmN}{Theorem}
%%\newtheorem{lemmaN}{Lemma}
%
%\theoremstyle{plain}
%\theorembodyfont{\upshape}
%\theoremsymbol{\ensuremath{\square}}
%%\newtheorem*{proof}{Proof}
%
%\theoremstyle{plain}
%\theorembodyfont{\upshape}
%\theoremsymbol{}
%\newtheorem{defn}[equation]{Definition}
%\newtheorem{conj}[equation]{Conjecture}
%\newtheorem{rem}[equation]{Remark}
%\newtheorem*{note}{Note}
%
%\theoremstyle{plain}
%%\theoremindent1\parindent
%%\theoremheaderfont{\kern-1\parindent\normalfont\bfseries} 
%\theoremsymbol{}
%\newtheorem{example}[equation]{Example}
%
%%\newcommand{\qedhere}{\qed}
%
%\newcommand{\xqed}[1][\ensuremath{\square}]{%
%  \leavevmode\unskip\penalty9999 \hbox{}\nobreak\hfill
%  \quad\hbox{\ensuremath{#1}}
%}
